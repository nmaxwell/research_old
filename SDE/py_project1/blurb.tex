\documentclass[12pt]{article}

\usepackage{amsmath}
\usepackage{amssymb}
\usepackage{amsfonts}

\usepackage{latexsym}
\usepackage{graphicx}
\usepackage{colonequals}

\setlength\topmargin{-1in}
\setlength{\oddsidemargin}{-0.5in}
%\setlength{\evensidemargin}{1.0in}

%\setlength{\parskip}{3pt plus 2pt}
%\setlength{\parindent}{30pt}
%\setlength{\marginparsep}{0.75cm}
%\setlength{\marginparwidth}{2.5cm}
%\setlength{\marginparpush}{1.0cm}
\setlength{\textwidth}{7.5in}
\setlength{\textheight}{10in}


\usepackage{listings}

\newcommand{\pset}[1]{ \mathcal{P}(#1) }
\newcommand{\partset}[1]{ \mathcal{P}^{*}(#1) }
\newcommand{\st}[0]{ \textrm{ s.t. } }
\newcommand{\fall}[0] { \textrm{ for all } }
\newcommand{\wrt}[0] { \textrm{ w.r.t. } }
\newcommand{\aew}[0] { \textrm{a.e.} }

\newcommand{\nats}[0] { \mathbb{N}}
\newcommand{\reals}[0] { \mathbb{R}}
\newcommand{\cmplxs}[0] { \mathbb{C}}
\newcommand{\complexes}[0] { \mathbb{C}}
\newcommand{\exreals}[0] {  [-\infty,\infty] }
\newcommand{\eps}[0] {  \epsilon }
\newcommand{\A}[0] { \mathcal{A} }
\newcommand{\B}[0] { \mathcal{B} }
\newcommand{\C}[0] { \mathcal{C} }
\newcommand{\D}[0] { \mathcal{D} }
\newcommand{\E}[0] { \mathcal{E} }
\newcommand{\F}[0] { \mathcal{F} }
\newcommand{\G}[0] { \mathcal{G} }
\newcommand{\M}[0] { \mathcal{M} }
\newcommand{\cS}[0] { \mathcal{S} }

\newcommand{\om}[0] { \omega }
\newcommand{\Om}[0] { \Omega }

\newcommand{\Bl}[0] { \mathcal{B} \ell }
\newcommand{\Ell}[0] { \mathcal{L} }

\renewcommand{\Re}{ \operatorname{Re} }
\renewcommand{\Im}{ \operatorname{Im} }

\newcommand{\IF}[0] { \; \textrm{if} \; }
\newcommand{\THEN}[0] { \; \textrm{then} \; }
\newcommand{\ELSE}[0] { \; \textrm{else} \; }
\newcommand{\AND}[0]{ \; \textrm{ and } \;  }
\newcommand{\OR}[0]{ \; \textrm{ or } \; }

\newcommand{\rimply}[0] { \Rightarrow }
\newcommand{\limply}[0] { \Leftarrow }
\newcommand{\rlimply}[0] { \Leftrightarrow }
\newcommand{\lrimply}[0] { \Leftrightarrow }

\newcommand{\rarw}[0] { \rightarrow }
\newcommand{\larw}[0] { \leftarrow }

\newcommand{ \defeq }[0] { \colonequals }
\newcommand{ \eqdef }[0] { \equalscolon }
%\newcommand{ \cf }[1] { \chi_{_{#1}} }
\newcommand{ \cf }[1] { \mathbf{1}_{#1} }

\newtheorem{thm}{Theorem}[section]


\begin{document}

Research notes, Dr. Bodmann, Nick Maxwell.

\addvspace{5pt} \hrule

\section*{Theory}

{\bf Theorem 1:} (Girsanov) Let $W(t) \in \reals^n$ be a Brownian motion with respect to the measure $P$,  and $X(t) \in \reals^n$ an It\^o process given by 

$$
dX(t) = a(t,\om) dt + dW(t), \; \; 0 \le t \le T \le \infty.
$$

\noindent
Then with

$$
M(t, \om ) = \exp \left( - \int_0^t a(s,\om) \cdot dW(s) - \frac{1}{2} \int_0^t a^2(s,\om) \, ds  \right), \; \; 0 \le t \le T
$$

\noindent
define the measure $Q$ by 

$$
dQ(\om) = M(T, \om) dP(\om),
$$

\noindent
then $X(t)$ is an $N$-dimensional Brownian motion wirh respect to $Q$. \\



{\bf Theorem 2:} Let $D \subset \reals^n$, open and connected, $\partial D$ its boundary, and $f: D \rarw \reals$. Let $W(t)$, $X(t)$, $M(t)$, $P$, as in the last theorem. Define

$$
\tau_{x}(\om) = \inf \left( \left\{  0 \le t \le T;  X(t, \om) \not \in D,  \right\} \right),
$$

\noindent
where $X(0) = x$. Then

$$
u(x) = \int_\Om f ( X(\tau_x(\om), \om) ) \, M(\tau_x(\om), \om) \,  dP(\om) = E_{Q_x} [ f(X(\tau_x)) ],
$$

\noindent
wheres $dQ_x(\om) = M(\tau_x(\om), \om) dP(\om)$, solves the equation 

$$
\nabla^2 u(x) = \left \{ \begin{array}{cc} 
0, & x \in D \\
f(x), & x \in \partial D. \\
\end{array} \right.
$$

{\bf Remarks:}
The drift, $a(t,\om)$, may be chosen to increase the rate of convergence in the above integral.



\addvspace{5pt} \hrule

\section*{Experiment}








\end{document}


