\documentclass[12pt]{article}

\usepackage{amsmath}
\usepackage{amssymb}
\usepackage{amsfonts}
\usepackage{latexsym}
\usepackage{graphicx}

\setlength\topmargin{-1in}
\setlength{\oddsidemargin}{-0.5in}
%\setlength{\evensidemargin}{1.0in}

%\setlength{\parskip}{3pt plus 2pt}
%\setlength{\parindent}{30pt}
%\setlength{\marginparsep}{0.75cm}
%\setlength{\marginparwidth}{2.5cm}
%\setlength{\marginparpush}{1.0cm}
\setlength{\textwidth}{7.5in}
\setlength{\textheight}{10in}


\usepackage{listings}




\newcommand{\pset}[1]{ \mathcal{P}(#1) }


\newcommand{\nats}[0] { \mathbb{N}}
\newcommand{\reals}[0] { \mathbb{R}}
\newcommand{\exreals}[0] {  [-\infty,\infty] }
\newcommand{\eps}[0] {  \epsilon }
\newcommand{\A}[0] { \mathcal{A} }
\newcommand{\B}[0] { \mathcal{B} }
\newcommand{\C}[0] { \mathcal{C} }
\newcommand{\D}[0] { \mathcal{D} }
\newcommand{\E}[0] { \mathcal{E} }
\newcommand{\F}[0] { \mathcal{F} }
\newcommand{\G}[0] { \mathcal{G} }

\newcommand{\IF}[0] { \; \textrm{if} \; }
\newcommand{\THEN}[0] { \; \textrm{then} \; }
\newcommand{\ELSE}[0] { \; \textrm{else} \; }
\newcommand{\AND}[0]{ \; \textrm{ and } \;  }
\newcommand{\OR}[0]{ \; \textrm{ or } \; }

\newcommand{\rimply}[0] { \Rightarrow }
\newcommand{\limply}[0] { \Lefttarrow }
\newcommand{\rlimply}[0] { \Leftrightarrow }
\newcommand{\lrimply}[0] { \Leftrightarrow }

\begin{document}

\begin{flushleft}
Research Documentation - Stochastic Differential Equations Applied to the Linear Wave Equation
Nicholas Maxwell; Dr. Bodmann\\
\end{flushleft}

\begin{flushleft}
\addvspace{5pt} \hrule
\end{flushleft}	



\section*{Part 1 - Preliminaries}

\begin{flushleft}
\underline{Pushforward measure:}
\end{flushleft}

\begin{flushleft}
Given a measure space, $(X,\A,\mu)$, a measurable space, $(Y,\B)$, and an $(\A,\B)-$measurable function, $f:X \rightarrow Y$, we may construct a measure on $(Y,\B)$,  $f_*(\mu) := \mu \circ f^{-1}$.\\
pf: (1) $\phi \in \B$, $f_*(\mu) (\phi) = \mu(F^{-1}(\phi)) = \mu(\phi) = 0$. \\
(2) $\{ B_k \}_{k \in \nats } \in \B $, disjoint, $B := \cup_{k \in \nats} B_k$. $A := f^{-1}(B) = \cup_{k \in \nats} A_k $, $A_k := f^{-1}(B_k)$. Then $A \in \A$ by $f$ being $(\A,\B)-$measurable, and $\{ A_k \}_{k \in \nats }$ is disjoint because, when $E_1 \cap E_2 = \phi$, $E_1, E_2 \in \B$, $ \phi = f^{-1}(E_1 \cap E_2) = f^{-1}(E_1) \cap f^{-1}(E_2)$. Then, $f_*(\mu) ( \cup_{k \in \nats} B_k )$  = $\mu( f^{-1}( \cup_{k \in \nats} B_k ) )$ = $\mu( \cup_{k \in \nats} f^{-1}( B_k ) )$ = $\mu( \cup_{k \in \nats} A_k )$ = $ \sum_{k \in \nats } \mu( A_k ) $ = $ \sum_{k \in \nats } (f_*(\mu))( B_k ) $, by countable additivity of $\mu$. So $ (Y,\B,f_*(\mu))$ is a well defined measure space obtained by pushing forward $\mu$ via $f$.
\end{flushleft}


\begin{flushleft}
\underline{Probability law of a Stochastic process:}
\end{flushleft}

\begin{flushleft}
Given $(\Omega,\F,P)$ a probability space, and $T$ an index set, $(S, \A)$ a measurable space. Then $X: T \times \Omega \rightarrow S$ is a stochastic process when the $t-$section of $X$, $X_t(\omega) := X(t,\omega)$ is $(\F,\A)-$measurable for all $t \in T$. Let $S^T = \{ g: T \rightarrow S \}$. \\
Each stochastic process, $X$ induces a function, $\Phi_X: \Omega \rightarrow S^T$ by $\Phi_X(\omega) := t \mapsto X(t,\omega)$, so $\Phi_X (\omega)$ is the $\omega-$section of $X$. We're interested in defining a measure on a suitable sigma algebra on $S^T$, by pushing forward $P$ via $\Phi_X$.
\end{flushleft}




\end{document}


















